\documentclass[german]{article}

\usepackage{url}
\usepackage[german]{}   % enables umlaute
\usepackage[utf8]{inputenc}  % set encoding to utf8 otherwise no umlaute
\usepackage{graphicx}        % for including graphics
\usepackage[hidelinks]{hyperref}        % for using hyperlinks in the document
\usepackage{tabularx}        % for extending tabluar
\usepackage{multirow}        % for rowspan in tabularx
\usepackage{ltablex}         % tables over multiple pages
\usepackage{textcmds}        % for quote support
\usepackage{pdfpages}        % for pdf include
\usepackage{caption}
\usepackage{minted}
\usepackage{listings}
\usepackage{minted}
\usepackage[left=1.0in, right=1.0in, top=1.0in, bottom=1.0in]{geometry} % for custom page layout

\newenvironment{code}{\captionsetup{type=listing}}{}

\newmintedfile[yamlFile]{yaml}{
	linenos=true, 
	frame=single, 
	breaklines=true, 
	tabsize=2,
	numbersep=5pt,
	xleftmargin=10pt,
	baselinestretch=1,
	fontsize=\footnotesize
}

\newmintedfile[javaFile]{java}{
	linenos=true, 
	frame=single, 
	breaklines=true, 
	tabsize=2,
	numbersep=5pt,
	xleftmargin=10pt,
	baselinestretch=1,
	fontsize=\footnotesize
}

\newmintinline[inlineBash]{bash}{}

% Title Page
\title{CI/CD mit Openshift}
\author{Thomas Herzog B.Sc / Phillip Wurm B.Sc}


\begin{document}
\maketitle

\begin{abstract}
Dieses Dokument beinhaltet die Dokumentation der \emph{CI/CD}-Umgebung, die in Oepnshift aufgesetzt wurde.

\end{abstract}

\section{Entwickler Setup}
\label{sec:dev-setup}
Dieser Abschnitt beschreibt das Aufsetzen einer lokalen Entwicklungsumgebung für die Entwicklung mit Openshift. Es wird davon ausgegangen, dass auf einem Linux System gearbetiet wird.

\subsection{Openshift Origin}
Dieser Abschnitt beschreibt das Einrichten des lokalen Oepnshift \emph{Clusters}. Diese Beschreibung ist für Linux und es wird angenommen, das eine aktuelle Version von Docker installiert ist.
\newline
\newline
Es werden folgende Ressourcen benötigt, die unter den folgenden \emph{URLs} heruntergeladen werden können.
\begin{enumerate}
	\item\emph{\url{https://developers.redhat.com/products/openshift/download/}}:
	\newline
	Es wird die Version \emph{3.5.5.31.24} benötigt, die unter \emph{älter Versionen} gefunden werden kann. Mit dieser Binary wird der lokale \emph{Cluster} betrieben. 
	\newline
	Es wird ein aktiver JBoss \emph{Developer Account} vorausgesetzt.
	\item\emph{\url{https://github.com/openshift-evangelists/oc-cluster-wrapper/releases/tag/0.9.3}}:
	\newline
	Bei \emph{oc-cluster-wrapper} handlet es sich um ein Shell-Skript, welches das Arbeiten mit oc erleichtert.
\end{enumerate}
Das \emph{oc} Binary sowie das Skript \emph{oc-cluster-wrapper} müssen in den \emph{PATH} mitaufgenommen werden. Das Skript \emph{oc-cluster-wrapper} verwendet das \emph{oc} Binary, das mit \emph{oc} über den \emph{PATH} angesprochen werden kann.

\subsubsection{Allgemeine Tasks}
Dieser Abschnitt beschreibt die allgemeinen Tasks, die mit \emph{oc} und \emph{oc-cluster-wrapper} durchgeführt werden.

\begin{minted}{bash}
# Create or start persistent profile for local cluster named 'ci'
oc-cluster-wrapper up ci

# Stop the local cluster of profile 'ci'
oc-cluster-wrapper down ci

# Delete profile ci and all related cluster data
oc-cluster-wrapper destroy ci
\end{minted}id




\end{document}          
