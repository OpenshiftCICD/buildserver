\section{Entwickler Setup}
\label{sec:dev-setup}
Dieser Abschnitt beschreibt das Aufsetzen einer lokalen Entwicklungsumgebung für die Entwicklung mit Openshift. Es wird davon ausgegangen, dass auf einem Linux System gearbetiet wird.

\subsection{Openshift Origin}
Dieser Abschnitt beschreibt das Einrichten des lokalen Oepnshift \emph{Clusters}. Diese Beschreibung ist für Linux und es wird angenommen, das eine aktuelle Version von Docker installiert ist.
\newline
\newline
Es werden folgende Ressourcen benötigt, die unter den folgenden \emph{URLs} heruntergeladen werden können.
\begin{enumerate}
	\item\emph{\url{https://developers.redhat.com/products/openshift/download/}}:
	\newline
	Es wird die Version \emph{3.5.5.31.24} benötigt, die unter \emph{älter Versionen} gefunden werden kann. Mit dieser Binary wird der lokale \emph{Cluster} betrieben. 
	\newline
	Es wird ein aktiver JBoss \emph{Developer Account} vorausgesetzt.
	\item\emph{\url{https://github.com/openshift-evangelists/oc-cluster-wrapper/releases/tag/0.9.3}}:
	\newline
	Bei \emph{oc-cluster-wrapper} handlet es sich um ein Shell-Skript, welches das Arbeiten mit oc erleichtert.
\end{enumerate}
Das \emph{oc} Binary sowie das Skript \emph{oc-cluster-wrapper} müssen in den \emph{PATH} mitaufgenommen werden. Das Skript \emph{oc-cluster-wrapper} verwendet das \emph{oc} Binary, das mit \emph{oc} über den \emph{PATH} angesprochen werden kann.

\subsubsection{Allgemeine Tasks}
Dieser Abschnitt beschreibt die allgemeinen Tasks, die mit \emph{oc} und \emph{oc-cluster-wrapper} durchgeführt werden.

\begin{minted}{bash}
# Create or start persistent profile for local cluster named 'ci'
oc-cluster-wrapper up ci

# Stop the local cluster of profile 'ci'
oc-cluster-wrapper down ci

# Delete profile ci and all related cluster data
oc-cluster-wrapper destroy ci
\end{minted}id



