\section{\emph{Secrets}}
\label{sec:buildserver-secrets}
Dieser Abschnitt behandelt die verwendeten \emph{Secrets}.\\

Das \emph{Secret-Objekt} oder kurz \emph{Secret} bietet eine Möglichkt zum Speichern vertraulicher Informationen wie Passwörter, Konfigurationsdateien, dockercfg-Dateien, Anmeldeinformationen für Source-Repositorys und viele mehr. \emph{Secrets} entkoppeln sensible Inhalte von den Pods und können mithilfe eines Volumen-Plug-Ins in Containern bereitgestellt werden.

\begin{listing}[H]
	\centering
	\begin{minted}{yaml}
	apiVersion: "v1"
	kind: "Secret"
	metadata:
	  name: "test-secret"
	  namespace: "my-namespace"
	data: 
	  username: "dmFsdWUtMQ0K"
	  password: "dmFsdWUtMg0KDQo="
	stringData: 
	  hostname: "myapp.mydomain.com"
	\end{minted}
	\caption{Secret Definition}
\end{listing}

\subsection{Eigenschaften von Secrets}

Secrets können unabhängig von ihrer Definition referenziert werden. Sie werden in temporären \emph{File-storage facilities} (tmpfs) gespeichert und werden niemals auf einem Knoten abgelegt. Secrets können auch innerhalb eines Namensraums geteilt werden.

\subsection{Projekt Secrets}

In diesem Projekt werden in den diversen Builds verschiedene Secrets verwendet. Zu diesen Secrets zählen unter Anderem ein SSH-Key für GitHub und die jeweiligen Benutzernamen und Passwörter für zweichen OpenShift, Jenkins und Nexus.