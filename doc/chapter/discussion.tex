\section{Diskussion}
\label{sec:discussion}
Dieser Abschnitt behandelt die Diskussion, des implementierten Prototypen. Die implementierte CI/CD Umgebung in Openshift war relativ einfach zu realisieren, da Jenkins und Jenkins Pipeline von Openshift nativ unterstützt werden. Da Openshift eine relativ alte Version von Jenkins und seinen Plugins zur Verfügung stellt, wurde ein eigenes Jenkins Image über einen S2I \emph{Build} definiert, das eine aktuelle Version von Jenkins und seinen Plugins zur Verfügung stellt. Das war einfach zu relaisieren und in Openshift zu integrieren, da beim Anlegen einer Jenkins Pipeline nach einem bestehenden Service mit Namen \emph{jenkins} gesucht wird, um in dieser Jenkins Instanz den Pipeline \emph{Build} anzulegen. \\

Für Jenkins und Nexus werden auch Openshift \emph{Templates} bereitgestellt, die aber angepasst wurden, um unseren Wünschen zu entsprechen. Hier verhält es sich wie bei \emph{Dockerfiles}, bei denen man auch über kurz oder lang seine eigenen \emph{Dockerfiles} implementiert anstatt die zur Verfügung gestellten zu verwenden.\\

Wenn man die Grundkonzepte, die hinter Kubernetes und Openshift stehen, verstanden hat, ist es relativ leicht mit Opensift zu arbeiten und Strukturen wie eine CI/CD Umgebung zu realisieren. Vor allem die Richtlinien wie \emph{Dockerfiles} implementiert werden sollen sind essentiell. Wenn man sich z.B. mit \inlineBash{oc rsh} in einem Docker \emph{Container} via ssh verbindet, so ist man immer ein Benutzer mit einer zufälligen \emph{uid} in der Gruppe \emph{0}, daher müssen alle Rechte so gesetzt werden, dass ein Benutzer in dieser Gruppe auch die nötigen Rechte hat. \\

Openshift verlangt auch dass immer explizit ein Benutzer in der \emph{Dockerfile} angegeben wird, mit dem der Prozess laufen soll. Standardmäßig ist es auch nicht erlaubt einen Docker Container mit \emph{Root}-Rechten zu starten. In den beiden vorherig beschriebenen Fällen wird von Openshift der Docker Container mit einer zufälligen \emph{uid} gestartet, was dazu führen kann, dass der Docker \emph{Container} nicht starten kann.
